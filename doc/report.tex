\documentclass[12pt,a4paper,oneside,article]{article}

\usepackage[italian]{babel}
\usepackage[T1]{fontenc}
\usepackage[utf8]{inputenc}
\usepackage[margin=1in]{geometry} % narrower margins
\usepackage{amsmath}
\usepackage{hyperref}
\usepackage{pgf-umlcd} % add "simplified" to hide empty parts in UML
\usepackage{pgf-umlsd}
\usepackage{tikz}
\usepackage{tikz-uml}
\usepackage{enumitem}

\newlist{legal}{enumerate}{10}
\setlist[legal]{label*=\arabic*.}

\tikzumlset{fill usecase=white} % for use case diagram

\usetikzlibrary{shapes.multipart} % for state diagram

\title{Aquarium Simulator}
\author{
	\href{mailto:filippo.barbari@studio.unibo.it}{Elisa Albertini},
	\href{mailto:filippo.barbari@studio.unibo.it}{Filippo Benvenuti},
	\href{mailto:filippo.barbari@studio.unibo.it}{Filippo Barbari},
	\href{mailto:filippo.barbari@studio.unibo.it}{Emanuele Lamagna}
}
\date{2 ottobre 2022}

\begin{document}
	\maketitle
	
	\begin{abstract}
		Brief summary. (non so se ci server)
	\end{abstract}
	
	\tableofcontents
	\clearpage

    \section{Introduzione}
    
    \section{Processo di sviluppo}
    
    \section{Analisi dei requisiti}
    \subsection{Requisiti di business}
    \begin{legal}[label*=1.\arabic*.]
   		\item Simulazione dell'evoluzione dell'ecosistema di un acquario popolato di pesci e alghe che interagiscono tra loro
   		\begin{legal}
   			\item Generazione di diversi scenari configurabili
   			\item Interazione con la simulazione per la modifica e l’aggiunta di ulteriori parametri o vincoli
   		\end{legal}
    \end{legal}

	\subsection{Requisiti utente}
	\begin{legal}[label*=2.\arabic*.]
    	\item osservare in tempo reale le interazioni tra le varie specie e il loro ciclo di vita, aggiungendo e rimuovendo pesci
    	\item i dati di ogni pesce verranno memorizzati all’interno di un database
    	\item si potranno estrarre informazioni di vario genere come la probabilità di sopravvivenza di una data specie
    	\item verranno visualizzati in tempo reale
    	\begin{legal}
    		\item numero di esseri viventi divisi per specie
    		\item temperatura dell’acqua
    		\item luce
    		\item pH dell’acqua
    		\item impurit\'a dell’acqua
    		\item ossigeno
    	\end{legal}
    	\item interagire con la simulazione regolando manualmente
    	\begin{legal}
    		\item aggiungendo cibo
    		\item regolando il termostato
    		\item regolando il filtro dell’acqua
    		\item pulendo l’acquario
    		\item regolare l’illuminazione per avere pi\'u o meno luce
    	\end{legal}
    	\item pesce controllabile manualmente dall’utente
    	\item estrazione dell’albero genealogico di ogni pesce dal database
    \end{legal}
    
    \subsection{Requisiti funzionali}
    \begin{legal}[label*=3.\arabic*.]
    	\item La simulazione è basata su un modello ad eventi discreti in cui è presente un tempo virtuale che scandisce le iterazioni
    \end{legal}

	\subsection{Requisiti non funzionali}
	\begin{legal}[label*=4.\arabic*.]
		\item deve essere fast
	\end{legal}
    
    \subsection{Requisiti di implementazione}
    \begin{legal}[label*=5.\arabic*.]
    	\item monadi
    	\item flyweight (tipo un sacco)
    \end{legal}
    
    \section{Design dell'architettura}
    \subsection{Casi d'uso}
    \begin{tikzpicture}
    	\begin{umlsystem}[x=5] {} % empty title
    		\umlusecase[name=a,width=2.5cm] {Start/stop/save/load simulation}
    		\umlusecase[name=b,x=6,width=2.5cm] {Use case b}
    		\umlusecase[name=c,x=6,y=-3,width=2.5cm] {Use case c}
    		\umlusecase[name=d,y=-3,width=2.5cm] {Use case d}
    	\end{umlsystem}
    
    	\node [above] at (current bounding box.north) {Aquarium Simulator};
    	
    	\umlactor[y=-1] {User}
    	%\umlactor[y=-3] {Actor 2}
    	%\umlactor[x=15] {Actor 3}
    	
    	\umlassoc{User}{a}
    	%\umlassoc{Actor 2}{a}
    	
    	
    	% \umlextend{a}{b}
    	% \umlinclude{c}{d}
    	
    	% manual versions of the above
    	\draw [tikzuml dependency style] (a) -- node[above] {$\ll \text{extend} \gg$} (b);
    	\draw [tikzuml dependency style] (d) -- node[above] {$\ll \text{include} \gg$} (c);
    	
    	% bent association
    	%\draw [tikzuml association style] (a) to[bend right=10] (Actor 3);
    \end{tikzpicture}
    
    \section{Design dettagliato}
    (serve che sia una sezione separata dal design architetturale)
    \subsection{Class diagram}
    \iffalse
    \begin{tikzpicture}
    	\begin{class}[text width=5cm]{BankAccount}{0,0}
    		\attribute{owner : String}
    		\attribute{balance : Dollars = 0}
    		\operation{deposit ( amount : Dollars )}
    		\operation[0]{withdrawl ( amount : Dollars )}
    	\end{class}
    	\begin{class}[text width=7cm]{CheckingAccount}{-5,-5}
    		\inherit{BankAccount}
    		\attribute{insufficientFundsFee : Dollars}
    		\operation{processCheck ( checkToProcess : Check )}
    		\operation{withdrawal ( amount : Dollars )}
    	\end{class}
    	\begin{class}[text width=7cm]{SavingsAccount}{5,-5}
    		\inherit{BankAccount}
    		\attribute{annualInteresRate : Percentage}
    		\operation{depositMonthlyInterest()}
    		\operation{withdrawal ( amount : Dollars)}
    	\end{class}
    \end{tikzpicture}\fi
    
    \subsection{State diagram}
    \iffalse
    \begin{tikzpicture}[stateNode/.style={rectangle split, rectangle split parts=2, draw, rounded corners, fill=yellow!10}]
    	\node[stateNode]{
    		\tikz\node[draw=red, rectangle, rounded corners]{title};
    		\nodepart{two}
    		\begin{tabular}{c}
    			content \\ more content
    		\end{tabular}
    	};
    \end{tikzpicture}\fi
    
    \subsection{Sequence diagram}
    \iffalse
    \begin{sequencediagram}
    	\newthread[blue]{s1}{:Server1}
    	\newinst{db}{:Database}
    	\newthread[red]{s2}{:Server2}
    	\begin{call}{s1}{reading}{db}{data}
    		\postlevel
    	\end{call}
    	\prelevel\prelevel
    	\setthreadbias{east}
    	\begin{call}{s2}{reading}{db}{data}
    		\postlevel
    	\end{call}
    \end{sequencediagram}\fi
    
    \section{Implementazione}
    
    \section{Conclusioni e lavori futuri}
\end{document}