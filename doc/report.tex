\documentclass[12pt,a4paper,oneside,article]{article}

\usepackage[italian]{babel}
\usepackage[T1]{fontenc}
\usepackage[utf8]{inputenc}
\usepackage[margin=1in]{geometry} % narrower margins
\usepackage{amsmath}
\usepackage{hyperref}
\usepackage{pgf-umlcd} % add "simplified" to hide empty parts in UML
\usepackage{pgf-umlsd}
\usepackage{tikz}
\usepackage{tikz-uml}
\usepackage{enumitem}

\newlist{legal}{enumerate}{10}
\setlist[legal]{label*=\arabic*.}

\tikzumlset{fill usecase=white} % for use case diagram

\usetikzlibrary{shapes.multipart} % for state diagram

\title{Aquarium Simulator}
\author{
	\href{mailto:filippo.barbari@studio.unibo.it}{Elisa Albertini},
	\href{mailto:filippo.barbari@studio.unibo.it}{Filippo Benvenuti},
	\href{mailto:filippo.barbari@studio.unibo.it}{Filippo Barbari},
	\href{mailto:filippo.barbari@studio.unibo.it}{Emanuele Lamagna}
}
\date{\today}

\begin{document}
	\maketitle
	
	\tableofcontents
	\clearpage

    \section{Introduzione}
    
    \section{Processo di sviluppo}
    
    \section{Analisi dei requisiti}
    \subsection{Requisiti di business}
%    \begin{legal}[label*=1.\arabic*.]
%   		\item Simulazione dell'evoluzione dell'ecosistema di un acquario popolato di pesci e alghe che interagiscono tra loro
%   		\begin{legal}
%   			\item Generazione di diversi scenari configurabili
%   			\item Interazione con la simulazione per la modifica e l’aggiunta di ulteriori parametri o vincoli
%   		\end{legal}
%    \end{legal}
	 \begin{legal}[label*=1.\arabic*.]
	 	\item \textbf{Ambito}: Siamo in un negozio di pesci dove viene richiesta la simulazione di un acquario reale personalizzabile, inserendo quindi più o meno pesci di diverso tipo immersi in un ambiente più o meno ricco di vegetazione, con lo scopo di osservare la salute dei pesci nel corso della loro vita tramite interazioni tra loro e l'ambiente circostante.
	 	Il prodotto finale sarà utile sia per mantenere un acquario digitale all'interno del negozio, sia per essere venduto al cliente finale, il quale potrà utilizzarlo per preparasi al meglio al mantenimento dei pesci prima dell'effettivo acquisto. 
	 	\item \textbf{Esigenze aziendali}:
	 	\begin{legal}
	 		\item Simulazione di un acquario contenente pesci e vegetazione.
	 		\item Interfaccia grafica d'intuitiva interazione.
	 		\item Possibilità di visionare i pesci in tempo reale.
	 		\item Facile installazione su diversi dispositivi fissi (il mobile non è richiesto).
	 		\item Possibilità di visionare l'andamento dei parametri relativi ai pesci.
	 	\end{legal}
	 	\item \textbf{Problemi}: Il negozio ha ricevuto diverse richieste di rimborso a seguito del decesso di alcuni pesci venduti, non essendo in grado di dimostrare che questi fossero stati in ottima salute e che il vero motivo del fattaccio fosse un'erronea gestione dell'acquario, il negozio richiede un simulatore da fornire ai clienti per tutelarsi e tutelare i pesci.
	 \end{legal}
	\subsection{Requisiti utente}
	Con lo scopo di mantenere il più vero possibile la stesura dei requisiti d'utente, abbiamo simulato una vera e propria intervista con il Product Owner:\\
	
	{\large \textbf{Descriva in una frase ciò di cui ha bisogno.}}\\
	Buongiorno, sono Filippo Benvenuti, proprietario e commesso del negozio di pesci "lo pesciazzo", in poche parole ho bisogno di un programma che dia la possibilità ai miei clienti di provare, prima di comprarmi i pesci, a mantenere una sorta di acquario digitale che faccia da simulazione ad uno vero.
	\\
	
	{\large \textbf{Cosa intende lei per simulazione?}}\\
	Non sono un esperto, mi riferisco alla possibilità di portarsi a casa un acquario in tutto e per tutto che abbia la possibilità di contenere più o meno pesci e magari anche la vegetazione, per dare al cliente la possibilità di allenarsi prima con un acquario in cui non può fare del male a nessuno e poi quando è più esperto tornare in negozio e acquistare un vero acquario con veri pesci. In questo modo spero di diminuire il numero di decessi dei pesci che vendo, purtroppo...
	\\
	
	{\large \textbf{Quali sono le caratteristiche di un acquario?}}\\
	Un acquario in primo luogo si differenzia dagli altri per il volume d'acqua che riesce a contenere e la forma, ma non penso che questi siano dati essenziali per la versione digitale, mi immagino un acquario piatto dove i pesci si muovono solo a destra a sinistra in alto e in basso.. La dimensione dell'acquario potrebbe essere utile anche se influisce solo sul numero di pesci che riesce a contenere. Le caratteristiche principali che poi sono quelle che influenzano i pesci che ci vivono all'interno riguardano per la maggior parte l'acqua, parliamo ad esempio della temperatura dell'acqua, importante sia quella giusta per i tipi di pesci all'interno dell'acquario e la vegetazione, non è difficile esistono sistemi che dopo averli impostati inizialmente la mantengono fissa scaldandola quando serve, raro che l'acqua sia troppo calda. Il PH dell'acqua è particolare, alcuni pesci sono molto sensibili ai cambiamenti del PH, bisogna stare attenti che questo valore non diventi ne troppo alto ne troppo basso, la causa principale dei decessi dei pesci avviene per una scarsa o del tutto assente considerazione del PH, il quale tende ad alzarsi per effetto della presenza dei pesci, le piante possono aiutare ad abbassarlo. Le piante inoltre offrono l'importante ossigeno per i pesci, spesso succede che non è abbastanza e si usa una pompa che ne aumenta la concentrazione nell'acqua, sappiamo bene cosa succederebbe se venisse a mancare l'ossigeno. La luce nell'acquario è utile alle piante per crescere, i pesci non sono molto influenzati, lo sono invece dall'impurità dell'acqua, ecco perché è importante tenere sempre molto pulito l'acquario!
	\\
	
	{\large \textbf{E invece le caratteristiche di un pesce?}}\\
	Ogni pesce ha caratteristiche proprie che lo contraddistinguono, in realtà la complessità biologica dei pesci è tanta, per questo programmino mi limiterei a distinguerli in modo semplificato, d'altronde l'obiettivo è quello di formare clienti inesperti non di certo un biologo marino. Di prima battuta possiamo dividere i pesci in carnivori ed erbivori, entrambi se non nutriti inizieranno a cacciare per trovarsi il cibo in autonomia, i carnivori diventano violenti attaccando gli altri pesci, preferiscono gli erbivori se ci sono, gli erbivori oltre a scappare se malnutriti attaccheranno le piante nell'acquario, mangiandole. Le principali caratteristiche di un pesce possiamo semplificarle in dimensione, velocità, fame ed età, queste caratteristiche sono naturalmente legate tra di loro un pò come succede per gli umani, ad esempio un pesce di grandi dimensioni e età avanzata andrà sicuramente più piano di uno snello e giovincello, all'aumentare delle dimensioni aumenta anche la fame, per aumentare la veridicità dell'acquario sarebbe importante tenere in considerazione queste caratteristiche per vedere come influenzano la vita di un pesce durante la simulazione.
	\\
	
	{\large \textbf{In che modo l'acquario influisce sui pesci e viceversa?}}\\
	Brevemente, i pesci vengono influenzati dall'acquario secondo i parametri di cui abbiamo discusso prima: Se la temperatura è troppo alta osserveremo i pesci muoversi velocemente con tutte le conseguenze del caso, se troppo bassa il contrario, in entrambi i casi un'esposizione prolungata a temperature estreme li conduce alla morte. Il PH è subdolo, se fuori dal range ottimale porterà alla morte dei pesci quasi casualmente. Se l'ossigeno dovesse scarseggiare vedremo i pesci avvicinarsi alle piante per tentare di assorbirne il più possibile, anche in questo caso troppo poco ossigeno porterà al decesso. La luce non influisce sui pesci, certamente più luce c'è meglio li vedremo, ma attenzione alle piante! L'impurità dell'acqua rende l'ambiente viscoso, i pesci faticheranno a muoversi li vedremo quindi rallentare come se fosse freddo, questo fortunatamente non porta ad una morte improvvisa, ma un pesce troppo lento non riuscirà a nutrirsi, purtroppo causa ed effetto, la morte. I pesci influenzano l'acquario, tramite gli escrementi contribuiscono all'impurità dell'acqua, più pesci ci sono più in fretta l'acquario si sporcherà a parità di dimensione, inoltre i pesci consumano ossigeno, anche in questo caso la popolazione fa la differenza, fortunatamente i pesci non sono abbastanza voluminosi da interferire con la temperatura dell'acqua anche se ne soffrono i cambiamenti. In generale lo sporco nell'acqua aumenta il PH dell'acqua, effetto indiretto indesiderato dai pesci.
	\\
	
	{\large \textbf{Quali sono le caratteristiche della vegetazione? Come interagisce con il resto?}}\\
	La vegetazione ha un ruolo fondamentale nell'acquario, si distingue per numero e per altezza, una pianta più grande sarà bersaglio dei pesci più affamati, importantissima in quanto produce ossigeno e diminuisce il livello del PH nell'acqua, ciò non ci giustifica a riempire l'acquario di piante, questo porterebbe entrambi i livelli fuori dal range ammissibile causando ancora una volta la morte dei pesci. Come per i pesci potrebbe essere utile pensare di poter aggiungere e rimuovere le piante all'interno dell'acquario per mantenere il più possibile i parametri stabili, essendo questa un'operazione semplice, le piante non scappano, è sicuramente da tenere in considerazione. Le piante sono influenzate da solo il livello della luce nell'acquario, più luce significa più velocità di crescita, piante più grandi producono più ossigeno e depurano meglio l'acqua, ma vale lo stesso ragionamento catastrofico di prima, al contrario poca luce porterà le piante alla morte.
	\\
	
	{\large \textbf{Quali azioni può compiere il cliente su questo acquario digitale?}}\\
	Ovviamente il cliente dovrà interagire con l'acquario, altrimenti non potrebbe imparare niente e sarebbe inutile, in particolare in base a come abbiamo detto che l'ecosistema funziona, il proprietario dell'acquario potrà decidere di aggiungere o togliere pesci o piante, fornire cibo per erbivori o per carnivori, pulire l'acquario, regolare luce temperatura e ossigenazione dell'acqua, le basi per imparare a prendersi cura dei pesci. Per semplificare il lavoro di un principiante, si può pensare alla possibilità di mettere in pausa e conseguentemente di riprendere la simulazione, in questo modo in momenti critici si può imparare cosa fare con calma.
	\\
	
	{\large \textbf{Ha qualche nota da aggiungere?}}\\
	Per la prima versione del prodotto finito quello che ci siamo detti è più che sufficiente, ma per il futuro potrebbe essere interessante aggiungere qualche funzionalità per renderlo più apprezzabile come ad esempio: la possibilità di controllare un pesce all'interno dell'acquario manualmente per rendere la simulazione non solo un lavoro ma anche ludica, ricostruire l'albero genealogico dei pesci, per osservare la dinastia e rendersi conto di quale famiglia sia stata la più forte nelle generazioni. Inoltre si potrebbe consentire di spegnere l'acquario e riprendere dal punto in cui si era rimasti e mettere la possibilità di velocizzare o rallentare la simulazione per evitare le parti noiose o per osservare nel dettaglio certi momenti.
	\\
	
	Dall'intervista con Filippo Benvenuti il Product Owner, sono stati dedotti i seguenti requisiti d'utente:
	
	\begin{legal}[label*=2.\arabic*.]
    	\item osservare in tempo reale:
    	\begin{legal}
    		\item ciclo di vita dei pesci.
    		\item interazioni fra pesci.
    		\item interazioni fra pesci e vegetazione.
    	\end{legal}
    	\item aggiungere e rimuovere pesci in tempo reale.
    	\item memorizzazione dati dell'acquario all’interno di un database:
    	\begin{legal}
    		\item possibilità di estrarre informazioni interessanti.
    		\item visionare grafici sull'andamento dei dati salvati.
    	\end{legal}
    	\item visualizzazione in tempo reale di:
    	\begin{legal}
    		\item numero di esseri viventi divisi per specie.
    		\item temperatura dell’acqua.
    		\item luce.
    		\item pH dell’acqua.
    		\item impurit\'a dell’acqua.
    		\item ossigeno.
    	\end{legal}
    	\item interazione con la simulazione tramite:
    	\begin{legal}
    		\item aggiunta cibo.
    		\item regolazione termostato.
    		%\item sostituzione filtro dell’acqua.
    		\item pulizia acquario.
    		\item regolazione illuminazione.
    		\item play/stop della simulazione.
    	\end{legal}
    	\item Opzionali per il futuro:
    	\begin{legal}
    		\item pesce controllabile manualmente dall’utente.
    		\item estrazione dell’albero genealogico di ogni pesce dal database.
    		\item salvataggio/caricamento della simulazione.
    		\item velocizzare/rallentare la simulazione
    	\end{legal}
    \end{legal}
    
    \subsection{Requisiti funzionali}
    \begin{legal}[label*=3.\arabic*.]
    	\item GUI
    	\begin{legal}
    		\item controlli utente
    		\begin{legal}
    			\item controllo intensità luminosa
    			\item controllo temperatura
    			\item controllo filtro dell'acqua (ossigenazione)
    			\item aggiunta pesce o alga
    			\item rimozione pesce o alga
    			\item start/stop simulazione
    			\item aggiunta cibo per pesci
    			\item pulizia acquario
    		\end{legal}
    		\item simulation view
    		\item grafici andamento parametri
    		\item download dati simulazione (csv)
    		\item visualizzazione parametri in tempo reale
    		\item cronistoria eventi
    	\end{legal}
    	\item simulation engine
    	\begin{legal}
    		\item acquario
    		\begin{legal}
    			\item dimensione
    			\item parametri acqua
    			\item ecosistema (popolazione + vegetazione)
    		\end{legal}
    		\item pesce
    		\begin{legal}
    			\item posizione nell'acquario
    			\item parametri del pesce
    			\begin{legal}
    				\item età
    				\item velocità
    				\item fame
    				\item dimensione
    				\item nome
    			\end{legal}
    			\item movimento
    		\end{legal}
    		\item alga
    		\begin{legal}
    			\item posizione nell'acquario
    			\item dimensione
    			\item velocità produzione ossigeno
    		\end{legal}
    		\item interazioni fra entità
    		\begin{legal}
    			\item pesce $\iff$ pesce
    			\item pesce $\Rightarrow$ alga
    			\item pesce $\iff$ acquario
    			\item acquario $\iff$ alga
    		\end{legal}
%    		Amelia: dobbiamo muoverci
%    		Sempre Amelia: siamo super in ritardo (il progetto è iniziato ieri)
%    		ancora: pensavo di esserti simpatica
    	\end{legal}
    	\item database
    	\begin{legal}
    		\item memorizzazione dati ad ogni iterazione
    		\begin{legal}
    			\item dati pesci
    			\item dati acquario
    			\item dati alghe
    		\end{legal}
    		\item estrazione dati
    		\begin{legal}
    			\item csv
    			\item immagini
    		\end{legal}
    	\end{legal}
    \end{legal}

	\subsection{Requisiti non funzionali}
	\begin{legal}[label*=4.\arabic*.]
		\item GUI responsiva
		\item usabilità
		\item semplice installazione cross-platform
	\end{legal}
    
    \subsection{Requisiti di implementazione}
    \begin{legal}[label*=5.\arabic*.]
    	\item scala
%    	\begin{legal}
%    		\item monadi
%    		\item flyweight
%    	\end{legal}
    	\item scalafx
    	\item scala-test
    	\item prolog
    	\item git
    	\begin{legal}
    		\item github
    		\item github project
    		\item github action
    	\end{legal}
    	\item sbt
    	
    \end{legal}
    
    \section{Design dell'architettura}
    \subsection{Casi d'uso}
    \begin{tikzpicture}
    	\begin{umlsystem}[x=5] {} % empty title
    		\umlusecase[name=a,width=2.5cm] {Start/stop/save/load simulation}
    		\umlusecase[name=b,x=6,width=2.5cm] {Use case b}
    		\umlusecase[name=c,x=6,y=-3,width=2.5cm] {Use case c}
    		\umlusecase[name=d,y=-3,width=2.5cm] {Use case d}
    	\end{umlsystem}
    
    	\node [above] at (current bounding box.north) {Aquarium Simulator};
    	
    	\umlactor[y=-1] {User}
    	%\umlactor[y=-3] {Actor 2}
    	%\umlactor[x=15] {Actor 3}
    	
    	\umlassoc{User}{a}
    	%\umlassoc{Actor 2}{a}
    	
    	
    	% \umlextend{a}{b}
    	% \umlinclude{c}{d}
    	
    	% manual versions of the above
    	\draw [tikzuml dependency style] (a) -- node[above] {$\ll \text{extend} \gg$} (b);
    	\draw [tikzuml dependency style] (d) -- node[above] {$\ll \text{include} \gg$} (c);
    	
    	% bent association
    	%\draw [tikzuml association style] (a) to[bend right=10] (Actor 3);
    \end{tikzpicture}
    
    \section{Design dettagliato}
    (serve che sia una sezione separata dal design architetturale)
    \subsection{Class diagram}
    \iffalse
    \begin{tikzpicture}
    	\begin{class}[text width=5cm]{BankAccount}{0,0}
    		\attribute{owner : String}
    		\attribute{balance : Dollars = 0}
    		\operation{deposit ( amount : Dollars )}
    		\operation[0]{withdrawl ( amount : Dollars )}
    	\end{class}
    	\begin{class}[text width=7cm]{CheckingAccount}{-5,-5}
    		\inherit{BankAccount}
    		\attribute{insufficientFundsFee : Dollars}
    		\operation{processCheck ( checkToProcess : Check )}
    		\operation{withdrawal ( amount : Dollars )}
    	\end{class}
    	\begin{class}[text width=7cm]{SavingsAccount}{5,-5}
    		\inherit{BankAccount}
    		\attribute{annualInteresRate : Percentage}
    		\operation{depositMonthlyInterest()}
    		\operation{withdrawal ( amount : Dollars)}
    	\end{class}
    \end{tikzpicture}\fi
    
    \subsection{State diagram}
    \iffalse
    \begin{tikzpicture}[stateNode/.style={rectangle split, rectangle split parts=2, draw, rounded corners, fill=yellow!10}]
    	\node[stateNode]{
    		\tikz\node[draw=red, rectangle, rounded corners]{title};
    		\nodepart{two}
    		\begin{tabular}{c}
    			content \\ more content
    		\end{tabular}
    	};
    \end{tikzpicture}\fi
    
    \subsection{Sequence diagram}
    \iffalse
    \begin{sequencediagram}
    	\newthread[blue]{s1}{:Server1}
    	\newinst{db}{:Database}
    	\newthread[red]{s2}{:Server2}
    	\begin{call}{s1}{reading}{db}{data}
    		\postlevel
    	\end{call}
    	\prelevel\prelevel
    	\setthreadbias{east}
    	\begin{call}{s2}{reading}{db}{data}
    		\postlevel
    	\end{call}
    \end{sequencediagram}\fi
    
    \section{Implementazione}
    
    \section{Conclusioni e lavori futuri}
\end{document}